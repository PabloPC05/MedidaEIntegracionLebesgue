\section{Integración con respecto de una medida}
\subsection{Propiedades de las funciones medibles}
\begin{definición}[Medibilidad de funciones]
    Sea $(X, \mathcal{A})$ un espacio medible y sea $(Y, \mathcal{B})$ otro espacio medible. Una función $f: X \to Y$ es \textbf{medible} si para todo conjunto medible $B \in \mathcal{B}$, el conjunto inverso $f^{-1}(B) \in \mathcal{A}$.
\end{definición}
\begin{lema}
    Sea $\mathcal{A}$ una $\sigma$-álgebra y $f: (X, \mathcal{A}) \to (\mathbb{R}, \mathcal{B})$ es medible si y sólo si $f^{-1}((-\infty), c) \in \mathcal{A}$ para todo $c \in \mathbb{R}$. $f: (X, \mathcal{A}) \to (\overline{\mathbb{R}}, \overline{\mathcal{B}})$ es medible si y sólo si $f^{-1}((-\infty), c) \in \mathcal{A}$ para todo $c \in \mathbb{R}$ y $f^{-1}({\infty}), f^{-1}({-\infty}) \in \mathcal{A}$.
\end{lema}
\begin{proof}
    $\Rightarrow$: Es trivial, ya que si $f$ es medible, por definición $f^{-1}(B) \in \mathcal{A}$ para todo $B \in \mathcal{B}$ y en particular para $B = (-\infty, c)$.\\
    $\Leftarrow$: 
    Dados $a > b$ se tiene que: 
    \[ [a, b) = (- \infty, b) \setminus (- \infty, a), \quad (a, b) = \bigcup_{n \in \mathbb{N}} \left[a + \frac{b - a}{n +1}, b\right) \]
    Puesto que todo abierto en $\mathbb{R}$ es unión numerable de intervalos abiertos se cumple que $f^{-1}(U) \in \mathcal{A}$ para todo $U \in \tau$ la topología usual de $\mathbb{R}$. Esto significa que $\sigma(f^{-1}(\tau)) = f^{-1}(\sigma(\tau)) = f^{-1}(\mathcal{B}) \subset \mathcal{A}$ por lo que $f$ es medible. 
\end{proof}
\begin{teorema}
    Sean $f, g, f_n: (X, \mathcal{A}) \to (\mathbb{R}, \mathcal{B}), n \in \mathbb{N}$ funciones medibles con respecto a una $\sigma$-álgebra $\mathcal{A}$. Entonces:
    \begin{enumerate}
        \item La función $\varphi \circ f$ es medible para cualquier función de Borel $\varphi: \mathbb{R} \to \mathbb{R}$, en particular esto es cieto si $\varphi$ es continua.
        \item $\alpha f + \beta g$ es medible para todo $\alpha, \beta \in \mathbb{R}$.
        \item $f \cdot g$ es medible.
        \item Si $g(x) \neq 0$ para todo $x \in X$, entonces $\frac{f}{g}$ es medible.
        \item Si existe el límite $f_0(x) = \lim_{n \to \infty} f_n(x)$ para todo $x$, entones $f_0$ es medible. 
        \item Las funciones $\sup_{n} f_n$, $\inf_{n} f_n$, $\limsup_{n \to \infty} f_n$ y $\liminf_{n \to \infty} f_n$ son medibles.
    \end{enumerate}
\end{teorema}
\begin{proof}
    (I) Sea $B \in \mathcal{B}$, entonces $(\varphi \circ f)^{-1}(B) = f^{-1}(\varphi^{-1}(B)) \in \mathcal{A}$, luego $\varphi \circ f$ es medible.

    Por (I), para demostrar (II) basta con considerar el caso $\alpha = \beta = 1$ y observar que para $(-\infty, c) \in \mathcal{B}$,
    \begin{align*}
        (f + g)^{-1}((-\infty, c)) &= \{x \in X : f(x) + g(x) < c\} = \{x \in X : f(x) < c - g(x)\} \\
        &= \bigcup_{r \in \mathbb{Q}} (\{x \in X : f(x) < r\} \cap \{x \in X : r < c - g(x)\}).
    \end{align*}
    El lado derecho de esta relación pertenece a $\mathcal{A}$, puesto que las funciones $f$ y $g$ son medibles.

    (III) Dedúcese de la igualdad $fg = \frac{1}{4}[(f + g)^2 - f^2 - g^2]$ y de las afirmaciones ya probadas; en particular, el cuadrado de una función medible es medible por (I).

    (IV) Observando que la función $\varphi$ definida por
    \[
        \varphi(x) = \begin{cases}
            \frac{1}{x}, & x \neq 0, \\
            0, & x = 0,
        \end{cases}
    \]
    es de Borel (comprobarlo), obtenemos (IV).

    (V) Basta comprobar que
    \[
        \{x \in X : f_0(x) < c\} = \bigcup_{k=1}^{\infty} \bigcup_{n=1}^{\infty} \bigcap_{m=n+1}^{\infty} \left\{x \in X : f_m(x) < c - \frac{1}{k}\right\}.
    \]

    (VI) Basta observar que
    \[
        \overline{f}(x) = \lim_{n \to \infty} \max_{k=1,\ldots,n} f_k(x).
    \]
    y demostrar la medibilidad de $\max_{k=1,\ldots,n} f_k(x)$. Por inducción, esto se reduce al caso $n = 2$. En este caso, tenemos:
    \[
        \{x \in X : \max(f_1(x), f_2(x)) < c\} = \{x \in X : f_1(x) < c\} \cap \{x \in X : f_2(x) < c\}.
    \]
    La afirmación correspondiente para el ínfimo se verifica con la igualdad
    \[
        \underline{f}(x) = -\sup_{n \in \mathbb{N}}[-f_n(x)].
    \]
\end{proof}
\begin{lema}
    Sean $f_n: (X, \mathcal{A}) \to \mathbb{R}$ funciones medibles para cada $n \in \mathbb{N}$. Entonces, el conjunto $L$ de todos los puntos $x \in X$ tales que $\lim_{n \to \infty} f_n(x)$ existe, es finito y pertenece a $\mathcal{A}$. Lo mismo vale para los conjuntos $L^{-}$ y $L^{+}$ de todos aquellos puntos donde el límite es $-\infty$ o $+\infty$ respectivamente.
\end{lema}
\begin{proof}
    Basta observar que el punto $x$ está en $L$ si y sólo si $(f_n(x))_{n \in \mathbb{N}}$ es una sucesión de Cauchy, es decir, $\forall k \in \mathbb{N} \exists N \in \mathbb{N}$ tal que $|f_p(x) - f_q(x)| < \frac{1}{k}$ para todo $p, q \geq N$. Es decir, 
    $$L = \bigcap_{k = 1}^{\infty} \bigcup_{N = 1}^{\infty} \bigcap_{p, q \geq N} \{x : f_p(x) - f_q(x) < \frac{1}{k}\} \in \mathcal{A}$$
    Los casos de $L^{-}$ y $L^{+}$ se demuestran de forma análoga.
\end{proof}
\begin{definición}
    Una función $f: (X, \mathcal{A}) \to \mathbb{R}$ se dice que es simple si es combinación lineal (finita) de funciones características de conjuntos medibles. Esto es que $f$ viene dada por una familia de conjuntos medibles disjuntos dos a dos $\{A_k\}_{k \in \mathbb{N}} \subset \mathcal{A}$ y por coeficientes $\{\alpha_k\}_{k \in \mathbb{N}} \subset \mathbb{R}$, de modo que 
    $$f(x) = \sum_{k = 1}^{n} \alpha_k \chi_{A_k}(x)$$
\end{definición}
\begin{observación}
    La condición de que los conjuntos $A_k$ sean disjuntos y de que los coeficientes $\alpha_k$ sean distintos no es necesaria para la definición, pero garantiza que la expresión de $f = \sum_{k = 1}^{n} \alpha_k \chi_{A_k}$ es única salvo el orden de los sumandos.
\end{observación}
\begin{lema}
    Sea $f: (X, \mathcal{A}) \to \mathbb{R}$ medible y y limitada. Entonces existe una sucesión de funciones simples $(s_n)_{n \in \mathbb{N}}$ tales que $s_n \leq s_{n + 1} \leq f$ y $\lim_{n \to \infty} s_n(x) = f(x)$ para todo $x \in X$.
\end{lema}
\begin{proof}
    Sea $c = \sup_{x \in X}\{|f(x)| + 1\}$ Para cada $n \in \mathbb{N}$, particionamos el intervalo $[-c, c)$ en $n$ subintervalos disjuntos: 
    $$I_j = [-c + c(j - 2)2^{-n}, -c + cj2^{-n})$$
    de longitud $c2^{-n}$. Definamos entones $A_j = f^{-1}(I_j)$; está claro que $A_j \in \mathcal{A}$ ya que $\cup_{j = 1}^{n} A_j = X$. Sea $c_j$ el extremo izquierdo de $I_j$ y definamos la función simple $s_n(x) \leq s_{n + 1}$ y $s_n(x) \leq f(x)$ y también se cumple que 
    $$\sup_{x \in X} |f(x) - s_n(x)| \leq 2^{-n}c$$ 
    ya que la función $f$ lleva $A_j$ a $I_j$ y $s_n$ lleva $A_j$ a $c_j \in I_j$, que está a una distancia como máximo de $2^{-n}c$ de cualquier punto de $I_j$.
\end{proof}
\begin{corolario}
    Sea $f: (X, \mathcal{A}) \to \mathbb{R}$ medible. Entonces para toda función $\mathcal{A}$-medible, $f$ existe una sucesión de funciones simples $s_n$ que converge a $f$ en cada punto. Si $f \geq 0$ satisface además que $s_n \leq s_{n + 1} \leq f$ 
\end{corolario}
\begin{proof}
    Definamos las funciones $g_n$ por
    $$g_n(x) = \begin{cases}
        f(x), \text{ si } f(x) \in [-n, n] \\
        0 , \text{ en otro caso}
    \end{cases}$$
    Podemos encontrar funciones simples $s_n$ tales que $|s_n(x) - g_n(x)| < \frac{1}{n}$. Es claro que $\lim_{n \to \infty} g_n(x) = f(x)$ para todo $x \in X$. Si $f \geq 0$ por el lema anterior podemos escoger $s_n$ tales que $s_n \leq s_{n + 1} \leq f$. 
\end{proof}
\subsection{La integral de funciones positivas}
\begin{definición}[Integral de Lebesgue de funciones simples sobre un espacio de medidas]
    Sea $(X, \mathcal{A}, \mu)$ un espacio de medida. Para cualquier función simple $f: (X, \mathcal{A}, \mu) \to \mathbb{R}^+$, definimos su integral de Lesbesgue respecto a la medida $\mu$ como 
    $$\int_{X} f d\mu = \int_X f(x) d\mu(x) = \sum_{k = 1}^{n} c_k\mu(A_k)$$
    Si la medida que estamos utilizando es evidente (por ejemplo, en el caso de usar la medida de Lebesgue), podemos omitir la referencia a la medida anterior y escribir $\int_X f(x)$. Si $A \in \mathcal{A}$, entonces la integral de $f$ sobre el conjunto $A$ se define como la integral de la función simple $\chi_A f$, es decir, 
    $$\int_A f d\mu = \sum_{k = 1}^{n} c_k \mu(A_k \cap A)$$
    Cuando no se indique el conjunto de integración, se entenderá que se está integrando con respecto al dominio de $f$. 
\end{definición}
\begin{lema}[Propiedades de la integral de funciones simples]
    La integral de Lebesgue sobre funciones simples cumple las siguientes propiedades: 
    \begin{enumerate}
        \item Se verifica que
        $$\int_X f(x) d \mu(x) \leq \sup_{x \in X} f(x) \mu(X)$$
        \item Sea $\alpha, \beta \in \mathbb{R}$ entonces
        $$ \int_X [\alpha f(x) + \beta g(x)] d\mu(x) = \alpha \int_{X} f(x) d\mu(x) + \beta \int_{X} g(x)d\mu(x)$$
        En particular, si $A$ y $ B$ son conjuntos medibles disjuntos (en $\mathcal{A}$) se cumple que
        $$\int_{A \cup B} f(x) d\mu(x) = \int_A f(x) d\mu(x) + \int_B f(x) d\mu(x)$$
    \end{enumerate}
\end{lema}
\begin{proof}
    \begin{enumerate}
        \item Es evidente por la definición de la integral de funciones simples. Además la definición implica la igualdad 
        $$\int_{X} \alpha f(x) d\mu(x) = \alpha \int_{X} f(x) d\mu(x)$$
        Por lo tanto ahora basta con verificar la afirmación (2) para $\alpha = \beta = 1$. 
        \item Sea $f$ función simple que toma valores distintos $c_k$ tales que se definen $A_k = \{x \in X : f(x) = c_k\}$ y sea $g$ una función que toma valores distintos $b_j$ tales que se definen $B_j = \{x \in X : g(x) = b_j\}$. Entonces los conjuntos $A_k \cap B_j = \{x \in X : f(x) = c_k\} \cap \{x \in X : g(x) = b_j\} \subset \mathcal{A}$-disjuntos, entonces $f + g = c_k + b_j$ en el conjunto $A_k \cap B_j$. \\ Sea $\{d_r\}_{r = 1}^{l} = \{c_k + b_j\}_{k = 1, \ldots, n; j = 1, \ldots, m}$ y $C_r = \bigcup\{A_k \cap B_j : c_k + b_j = d_r\}$ para $r = 1, \ldots, l$. Entonces
        $$\int_{X} [f(x) + g(x)] = \sum_{r = 1}^{l} d_r \mu(C_r) = \sum_{r = 1}^{l} d_r \mu(\bigcup\{A_k \cap B_j : c_k + b_j = d_r\}) = $$
        $$ = \sum_{k = 1}^{n} \sum_{j = 1}^{m} (c_k + b_j) \mu(A_k \cap B_j) = \sum_{k = 1}^{n} c_k\mu(A_k) + \sum_{j = 1}^{m} b_j \mu(B_j) = \int_X f(x) d\mu(x) + \int_X g(x) d\mu(x)$$
        pues $\sum_{j = 1}^{m} \mu(A_k \cap B_j) = \mu(A_k)$ y $\sum_{k = 1}^{n}\mu(A_k \cap B_j) = \mu(B_j)$.
        Esta última afirmación se deduce de $\chi_{A \cup B} = \chi_A + \chi_B$, ya que los conjuntos $\{A_k\}$ y $\{B_j\}$ son disjuntos y cubren todo $X$. 
    \end{enumerate}
\end{proof}
\begin{corolario}
    Si $f$ y $g$ son funciones simples tales que $f \leq g$ casi en todo punto, entonces 
    $$\int_{X} f(x) d\mu(x) \leq \int_X g(x) d\mu(x)$$
\end{corolario}
\begin{proof}
    Sea $A = \{x \in X : f(x) \leq g(x)\}$. Entonces $A \in \mathcal{A}$ y $\mu(X \setminus A) = 0$. Entonces, por hipótesis tenemos que $g - f$ es una función simple en $A$. No obstante, la función podría ser negativa en $X \setminus A$, por lo que definamos la constante positiva:
    $$ c = \sup_{x \in X}\{|f(x) + g(x)|\}$$
    Por tanto definamos la función auxiliar 
    $$h(x) = g(x) - f(x) + c\chi_{X \setminus A}(x)$$
    Entonces $h$ es una función simple no negativa en $X$: 
    \begin{enumerate}
        \item Si $x \in A$: 
        $$h(x) = g(x) - f(x) + c \cdot 0 = g(x) - f(x) \geq 0$$
        \item Si $x \in X \setminus A$:
        $$h(x) = g(x) - f(x) + c \cdot 1 \geq -|g(x) - f(x)| + c \geq 0$$
    \end{enumerate}
    Ahora, por el lema anterior se cumple que 
    $$\int_X h(x) d\mu(x) = \int_X gd\mu(x) - \int_X f(x) d\mu(x) + \int_{X} c\chi_{X \setminus A}(x) d\mu(x) \geq 0$$
    Entonces, por la linealidad de la integral, tenemos que
    $$\int_X g(x) d\mu(x) - \int_X f(x) d\mu(x) \geq 0$$
\end{proof}
\begin{definición}[Integral de Lebesgue de funciones positivas sobre un conjunto medible]
    Sea $f: (X, \mathcal{A}) \to \mathbb{R}^+$ una función medible y $E \in \mathcal{A}$. Definimos la integral de Lesbesgue de $f$ respecto a $\mu$ sobre $E$ como 
    $$\int_E f d\mu = \sup\left\{\int_E s d\mu : s: X \to \mathbb{R}^+ \text{ medible }, s \leq f \right\} \in \mathbb{R}^+$$
    Al igual que para las funciones simples usaremos la notación $\int_E f \equiv \int_E f d\mu \equiv \int_E f(x) d\mu(x)$.
\end{definición}
\begin{lema}
    Sean $f, g : (X, \mathcal{A}, \mu) \to [0, \infty)$ medibles, $A, B, E \in \mathcal{A}$. Entonces se tienen las siguientes propiedades:
    \begin{enumerate}
        \item[(I)] Si $0 \leq f \leq g$, entonces
        $$\int_E f d\mu \leq \int_E g d\mu.$$
        \item[(II)] Si $A \subset B$ y $f \geq 0$, entonces
        $$\int_A f d\mu \leq \int_B f d\mu.$$
        \item[(III)] Si $f \geq 0$ y $c$ es una constante con $0 \leq c < \infty$, entonces
        $$\int_E cf d\mu = c \int_E f d\mu.$$
        \item[(IV)] Si $f(x) = 0$ para todo $x \in E$, entonces
        $$\int_E f d\mu = 0,$$
        incluso si $\mu(E) = \infty$.
        \item[(V)] Si $\mu(E) = 0$, entonces
        $$\int_E f d\mu = 0,$$
        aún si $f(x) = \infty$ para todo $x \in E$.
        \item[(VI)] Si $f \geq 0$, entonces
        $$\int_E f d\mu = \int_X \chi_E f d\mu.$$
    \end{enumerate}
\end{lema}
\begin{teorema}[Teorema de Convergencia Monótona de Lebesgue]
    Sea $(f_n)_{n \in \mathbb{N}}: (X, \mathcal{A}) \to \mathbb{R}$ una sucesión de funciones medibles en $X$, y supongamos que:
    \begin{enumerate}
        \item $0 \leq f_1(x) \leq f_2(x) \leq \ldots \leq \infty$ para todo $x \in X$. 
        \item $f_n(x) \to f(x)$ cuando $n \to \infty$ para todo $x \in X$
    \end{enumerate}
    Entonces $f$ es medible y se cumple que 
    $$\lim_{n \to \infty} \int_X f_n d\mu = \int_X f d\mu$$
\end{teorema}
\begin{proof}

    Como $\int f_n \leq \int f_{n+1}$ existe $\alpha \in \mathbb{R}^+$ tal que 
    $$\lim_{n \to \infty} \int_X f_n d\mu = \alpha$$

    Para ver la igualdad, dividiremos la demostración en las dos desigualdades: 
    \begin{enumerate}
        \item $\alpha \leq \int_X f d\mu$: Sea $f = \sup_{n \in \mathbb{N}} f_n$, entonces como $f$ es medible y $f_n \leq f$ se cumple que $\int f_n \leq \int f$ para todo $n$ por lo que tomando límites cuando $n \to \infty$ se tiene que $\alpha \leq \int_X f d\mu$.
        \item $\alpha \geq \int_X f d\mu$: Sea $s$ cualquier función simple medible tal que $0 \leq s \leq f$, sea $c \in (0,1)$ y definamos $E_n = \{x : f_n(x) \geq c s(x)\}$, $n \in \mathbb{N}$. Cada $E_n$ es medible, $E_n \subset E_{n + 1}$ y $X = \bigcup_{n = 1}^{\infty} E_n$. Para ver esta igualdad considérese $x \in X$, entonces se pueden dar dos casos: 
        \begin{enumerate}
            \item Si $f(x) = 0$ entonces $s(x) = 0$ (pues $s \leq f$) así que $cs(x) = 0 \leq f_1(x)$ luego, $x \in E_1$ 
            \item Si $f(x) > 0$ entonces $cs(x) < f(x)$ ya que $c < 1$ por lo que existe alguna $n$ tal que $x \in E_n$.
        \end{enumerate}
        También se cumple que para $n \in \mathbb{N}$ que $\int_X f_n \geq \int_{E_n} f_n \geq c \int_{E_n} s$ por lo que tomando el límite 
        $$\alpha \geq c \lim_{n \to \infty} \int_{E_n} s d\mu = c \int_X s$$
        donde la última igualdad se debe a la continuidad por abajo de la medida. Como $c \in (0,1)$ fue fijado arbitrariamente si tomamos $c \to 1^-$, obtenemos que $\alpha \geq \int_X s$. Como $s$ fue fijada arbitrariamente con $0 \leq s \leq f$, y por definición tenemos que 
        \[ \int_X f d\mu = \sup\left\{ \int_X s\,d\mu : s \text{ simple },\ 0 \leq s \leq f \right\} \]
        concluimos que $\alpha \geq \int_X f d\mu$.
    \end{enumerate}
\end{proof}
\begin{teorema}[Teorema de Beppo Levi]
    sea $(f_n)_{n \in \mathbb{N}}$ una sucesión de funciones medibles $f_n: X \to \mathbb{R}^+$ tal que 
    $$f(x) = \sum_{n = 1}^{\infty} f_n(x) \text{ para todo } x \in X$$
    Entonces $f$ es medible y se cumple que 
    $$\int_X f d\mu = \sum_{n = 1}^{\infty} \int_X f_n d\mu$$
\end{teorema}
\begin{proof}
    Primero, existen sucesiones crecientes $(s^1_n)_{n \in \mathbb{N}}$ y $(s^2_n)_{n \in \mathbb{N}}$ de funciones simples y medibles tales que $s_n \leq f_1$, $s_n \leq f_2$, $s^1_n \to f_1$ y $s^2_n \to f_2$. Definiendo $s_n = s^1_n + s^2_n$, observse que $s_n \to f_1 + f_2$. Entonces por el Teoema de la convergencia monótona anterior y por la linealidad de la integral en funciones simples, se tiene que 
    $$\int_X f_1 + f_2 d\mu = \int_X f_1 + \int_X f_2$$
    Sea $g_n = f_1 + \ldots + f_n$, entonces la sucesión $(g_n)_{n \in \mathbb{N}}$ crece monótonamente y converge a $f$. Aplicando la induccion observamos que 
    $$\int_X g_n d\mu = \sum_{k = 1}^{n} \int_X f_k d\mu$$
    Aplicando de nuevo el Teorema de la convergencia monótona se obtiene la igualdad deseada:
    $$\int_X f d\mu = \lim_{n \to \infty} \int_X g_n d\mu = \lim_{n \to \infty} \sum_{k = 1}^{n} \int_X f_k d\mu = \sum_{k = 1}^{\infty} \int_X f_k d\mu$$
\end{proof}
\begin{lema}[Lema de Fatou]
    Sea $\{f_n\}$ una sucesón de funciones medibles $f_n: X \to \mathbb{R}^*$ para cada entero positivo $n$. Entonces se cumple que
    $$\int_X \liminf_{n \to \infty} f_n d\mu \leq \liminf_{n \to \infty} \int_X f_n d\mu$$
\end{lema}
\begin{proof}
    Definamos $g_k(x) = \inf_{n \geq k} f_n(x) : k \in \mathbb{N}$. $x \in X$. Entocnes $g_k \leq f_k$, por lo que 
    $$\int_X g_k d\mu \leq \int_X f_k d\mu$$
    para todo $k \in \mathbb{N}$. Además, $0 \leq g_k \leq g_{k + 1}$, cada $g_k$ es medible y
    $$\lim_{k \to \infty} g_k(x) = \liminf_{n \to \infty} f_n(x)$$
    Por el teorema de la convergencia monótona, se tiene que
    $$\int_X \liminf_{n \to \infty} f_n d\mu = \int_X \lim_{k \to \infty} g_k d\mu = \lim_{k \to \infty} \int_X g_k d\mu \leq \liminf_{n \to \infty} \int_X f_n d\mu$$  
\end{proof}
\subsection{Integracion de funciones reales y complejas}
\begin{definición}
    Definimos $\mathcal{L}^1((X, \mathcal{A}, \mu), \mathbb{C}) = \mathcal{L}(X, \mathbb{C})$ como el conjunto de todas las funciones complejas medibles $f$ sobre $X$ tales que
    $$\int_X |f| d\mu < \infty$$
    Cabe señalar que la medibilidad de $f$ implica la de $|f|$, por lo tanto, la integral anterior está bien definida.\\
    Los elementos de $\mathcal{L}^1(X, \mathbb{C})$ se llaman funciones integrables en sentido de Lebesgue (respecto de $\mu$) o funciones sumables.
\end{definición}
\begin{definición}
    Sea $f = u + iv$ donde $u$ y $v$ son funciones reales medibles sobre $X$ y supongamos que $f \in \mathcal{L}^1(X, \mathbb{C})$. Definimos para todo conjunto medible $E$: 
    $$\int_E f d\mu = \int_E u^+ d\mu - \int_E u^- d\mu + i\left(\int_E v^+ d\mu - \int_E v^- d\mu\right)$$
    donde $g^+ = \max\{g, 0\}$ y $g^- = \max\{-g, 0\}$ para cualquier función real $g$. Las funciones $u^+, u^-, v^+, v^-$ son funciones reales medibles no negativas y sumables.. Además tenemos que $u^+ u^-, v^+ v^- \leq |f|$ por lo que cada una de estas cuatro integrales existe y es finita. 
\end{definición}
\begin{teorema}
    Sean $f, g \in \mathcal{L}^1(X, \mathbb{C})$ y $\alpha, \beta \in \mathbb{C}$. Entonces $\alpha f + \beta g \in \mathcal{L}^1(X, \mathbb{C})$ y
    $$\int_X [\alpha f(x) + \beta g(x)] d\mu(x) = \alpha \int_X f(x) d\mu(x) + \beta \int_X g(x) d\mu(x)$$
\end{teorema}
\begin{proof}
    La medibilidad de $\alpha f + \beta g$ es inmediata. Además, se tiene que
    $$\int_X |\alpha f(x) + \beta g(x)| d\mu(x) \leq |\alpha| \int_X |f(x)| d\mu(x) + |\beta| \int_X |g(x)| d\mu(x) < \infty$$
    Por lo tanto $\alpha f + \beta g \in \mathcal{L}^1(X, \mathbb{C})$. Finalmente queda probar que 
    $$\int_X f(x) + g(x) d\mu(x) = \int_X f(x) d\mu(x) + \int_X g(x) d\mu(x)$$
    y que 
    $$\int_X cf(x) d\mu(x) = c \int_X f(x) d\mu(x)$$
    Definamos la función auxiliar $h = f + g$, tenemos que 
    $$h^+ - h^- = f^+ - f^- + g^+ - g^-$$
    reordenado, esto es 
    $$h^+ + f^- + g^- = h^- + f^+ + g^+$$
    Así, por la linealidad de la integral para funciones no negativas, tenemos que 
    $$\int h^+ + \int f^- + \int g^- = \int f^+ + \int g^+ + \int h^-$$
    Y como cada una de estas integrales es finita, espejando obtenemos la primera igualdad.\\
    Para la segunda igualdad, sea $\alpha$ distinguamos casos: 
    \begin{enumerate}
        \item Si $\alpha \geq 0$, entonces el resultado es inmediato 
        \item Si $\alpha = -1$ entonces el resultado también es inmediato usando que $(-g)^+ = g^-$
        \item Si $\alpha = i$ el resultado también es sencillo: 
        $$\int (if) = \int i(u + iv) = \int (-v + iu) = \int -v + i \int u = i\int f$$
    \end{enumerate}
    Combinando estos resultados, obtenemos la segunda igualdad para $\alpha \in \mathbb{C}$ arbitrario.
\end{proof}
\begin{teorema}
    Sea $f \in \mathcal{L}^1(X, \mathbb{C})$. Entonces
    $$\left|\int_X f(x) d\mu(x)\right| \leq \int_X |f(x)| d\mu(x)$$
\end{teorema}
\begin{proof}
    Definamos $z = \int_X f(x) d\mu(x)$. Como $z$ es un número compejo, existe un número complejo $\alpha$ tal que 
    $$\alpha = \begin{cases}
        0 \text{ si } z = 0 \\
        \frac{z}{|z|} \text{ si } z \neq 0
    \end{cases} \implies |z| = \begin{cases}
        0 \text{ si } z = 0\\
        1 \text{ si } z \neq 0
    \end{cases}$$ y tal que $\alpha \cdot z = |z|$. Sea $u = \Re(\alpha f)$. Entonces $u \leq |\alpha f| = |f|$. Por tanto 
    $$\int_X f(x)d\mu = |z| = \alpha z = \alpha \int_X f(x) d\mu(x) = \int_X \alpha f(x) d\mu(x) = \int_X \Re(\alpha f(x)) d\mu(x) + i \int_X \Im(\alpha f(x)) d\mu(x) = $$
    $$ = \int u \leq \int |f|$$
    La parte imaginaria de la integral es nula ya que es igual a un módulo, i.e. a un número real. 
\end{proof}
\begin{teorema}[Teorema de la convergencia dominada de Lebesgue]
    Sea $(f_n)_{n \in \mathbb{N}}$ una sucesión de funciones complejas y medibles en $X$ tales que 
    $$f(x) = \lim_{n \to \infty} f_n(x) \text{ para todo } x \in X$$
    Si existe una función $g \in \mathcal{L}^1(X, \mathbb{C})$ tal que
    $$|f_n(x)| \leq g(x) \text{ para todo } x \in X \text{ y todo } n \in \mathbb{N}$$
    Entonces $f \in \mathcal{L}^1(X, \mathbb{C})$ y se cumple que
    $$\lim_{n \to \infty} \int_X |f_n(x) - f(x)| d\mu(x) = 0$$
    y además que 
    $$\lim_{n \to \infty} \int_X f_n(x) d\mu(x) = \int_X f(x) d\mu(x)$$
\end{teorema}
\begin{proof}
    Como $|f_n(x)| \leq g(x)$ para todo $n$ y $f_n(x) \to f(x)$, tomando límite obtenemos $|f(x)| \leq g(x)$. Como $f$ es medible (límite de funciones medibles) y $g \in \mathcal{L}^1(X, \mathbb{C})$, se tiene que $f \in \mathcal{L}^1(X, \mathbb{C})$.
    
    Además, observemos que:
    $$|f_n(x) - f(x)| \leq |f_n(x)| + |f(x)| \leq 2g(x)$$
    
    Como $f_n(x) \to f(x)$ para todo $x \in X$, tenemos que $|f_n(x) - f(x)| \to 0$ para todo $x \in X$, es decir:
    $$\liminf_{n \to \infty} |f_n(x) - f(x)| = 0 \text{ para todo } x \in X$$
    
    Aplicamos el Lema de Fatou a las funciones no negativas $h_n(x) = 2g(x) - |f_n(x) - f(x)| \geq 0$:
    $$\int_X \liminf_{n \to \infty} h_n \, d\mu \leq \liminf_{n \to \infty} \int_X h_n \, d\mu$$
    
    Como $\liminf_{n \to \infty} h_n(x) = 2g(x) - \liminf_{n \to \infty} |f_n(x) - f(x)| = 2g(x) - 0 = 2g(x)$, tenemos:
    $$\int_X 2g \, d\mu \leq \liminf_{n \to \infty} \int_X (2g - |f_n - f|) \, d\mu$$
    
    Por linealidad de la integral:
    $$\int_X 2g \, d\mu \leq \liminf_{n \to \infty} \left[\int_X 2g \, d\mu - \int_X |f_n - f| \, d\mu\right]$$
    
    Como $\int_X 2g \, d\mu < \infty$ (pues $g \in \mathcal{L}^1$), podemos escribir:
    $$\int_X 2g \, d\mu \leq \int_X 2g \, d\mu - \limsup_{n \to \infty} \int_X |f_n - f| \, d\mu$$
    donde usamos que $\liminf_{n \to \infty}(-a_n) = -\limsup_{n \to \infty}(a_n)$. Restando $\int_X 2g \, d\mu$ en ambos lados:
    $$0 \leq -\limsup_{n \to \infty} \int_X |f_n - f| \, d\mu$$
    
    Es decir:
    $$\limsup_{n \to \infty} \int_X |f_n - f| \, d\mu \leq 0$$
    
    Como siempre $\int_X |f_n - f| \, d\mu \geq 0$, tenemos:
    $$0 \leq \liminf_{n \to \infty} \int_X |f_n - f| \, d\mu \leq \limsup_{n \to \infty} \int_X |f_n - f| \, d\mu \leq 0$$
    
    Por tanto, existe el límite y:
    $$\lim_{n \to \infty} \int_X |f_n - f| \, d\mu = 0$$
    
    Finalmente, para la segunda conclusión:
    $$\left|\int_X f_n \, d\mu - \int_X f \, d\mu\right| = \left|\int_X (f_n - f) \, d\mu\right| \leq \int_X |f_n - f| \, d\mu \to 0$$
    
    Por tanto:
    $$\lim_{n \to \infty} \int_X f_n(x) \, d\mu(x) = \int_X f(x) \, d\mu(x)$$
\end{proof}
\subsection{Espacios $L^p$}
\begin{definición}
    Sean $p \in [1, \infty)$ y ($X$, $\mathcal{A}$, $\mu$) un espacio de medida. Definimos
    $$\mathcal{L}^p(X, \mathbb{F}) \equiv \mathcal{L}^p(X, \mu, \mathbb{F}) = \{f: X \to \mathbb{F} : f \text{ medible }, \|f\|_p < \infty\}$$
    donde $\|\cdot \|_p$ se define como 
    $$\|f\|_p = \left(\int_X |f(x)|^p d\mu(x)\right)^{1/p} \in [0, \infty)$$
    Definimos además $\mathcal{L}(X, \mathbb{F}) = \mathcal{L}^p(F, \mathbb{F})|_{\sim}$ 
    donde $f \sim g \iff \mu(\{x \in X : f(x) \neq g(x)\}) = 0$. Observese que si $f, g \in \mathcal{L}^p(X, \mathbb{F})$ y $f \sim g$ entonces $\|f\|_p = \|g\|_p$. \\
    Definimos también
    $$\mathcal{L}^{\infty}(X, \mathbb{F}) = \{f: X \to \mathbb{F} : f \text{ medible }, \|f\|_{\infty} < \infty\}$$
    donde 
    $$\|f\|_{\infty} = \inf\{M \geq 0 : |f(x)| \leq M \text{ para casi todo } x \in X\}$$
    y por definición $\inf\{\emptyset\} = \infty$. 
\end{definición}
\begin{definición}[Exponentes conjugados]
Sean $p,q\in[1,\infty]$. Decimos que $p$ y $q$ son exponentes conjugados (o duales) si
\[
    \frac{1}{p}+\frac{1}{q}=1.
\]
Para $1<p<\infty$ esto determina $q=\frac{p}{p-1}$. Se adoptan las convenciones
$p=1$ corresponde a $q=\infty$ y $p=\infty$ corresponde a $q=1$.
\end{definición}
\begin{proposición}[Desigualdad de Young]
    Sea $p \in (1, \infty)$, entonces para $a, b \in \mathbb{R}^+$ se tiene que
    $$ab \leq \frac{a^p}{p} + \frac{b^{p^*}}{p^*}$$
\end{proposición}
\begin{proof}
    Distingamos casos: 
    \begin{enumerate}
        \item Si $ab = 0$, la desigualdad es evidente
        \item Si $a, b > 0$, podemos escribir $a = tb$ con $t > 0$. Para demostrar que la desigualdad se cumple para todo $t$, basta ver que la función 
        $$f(t) = \frac{(tb)^p}{p} + \frac{b^{p^*}}{p^*} - (tb)b \geq 0 \quad \forall t \in \mathbb{R}^+$$
        Derivemos la función anterior 2 veces para analizar su crecimiento: 
        $$f^{\prime}(t) = b(tb)^{-1} - b^2, \quad f^{\prime \prime}(t) = (p - 1)b^2 (tb)^{p - 2} > 0$$
        Luego $f$ es estrictamente convexa. Aemás $f^{\prime}$ sólo se anula en $t_0 = b^{\frac{2 - p}{p - 1}}$, donde alcanza un mínimo relativo con $f(t_0) = 0$, luego $f(t) \geq 0$ para todo $t \in \mathbb{R}^+$.
    \end{enumerate}
\end{proof}
\begin{proposición}[Desigualdad de Hölder]
    Sean $p \in [1, \infty]$, $(X, \mu$)-espacio de medida y $f, g: X \to \mathbb{F}$ medibles. Entonces
    $$\|f g\|_1 \leq \|f\|_p \|g\|_q$$
\end{proposición}
\begin{proof}
    Distingamos casos:
    \begin{enumerate}
        \item Si $f \cdot g$ = 0, la desigualdad es evidente.
        \item Si $f \cdot g \neq 0$, aplicamos la desigualdad de Young a $a = |f(x)|/\|f\|_p$ y $b = |g(x)|/\|g\|_{p^*}$ para todo $x \in X$:
        $$\frac{\|fg\|_1}{\|f\|_p \|g\|_{p^*}} = \int \frac{|f|}{\|f\|_p} \frac{|g|}{\|g\|_{p^*}} \, \mathrm{d}\mu \le \int \frac{|f|^p}{p \|f\|_p^p} \, \mathrm{d}\mu + \int \frac{|g|^{p^*}}{p^* \|g\|_{p^*}^{p^*}} \, \mathrm{d}\mu = \frac{1}{p} + \frac{1}{p^*} = 1$$
        por lo que $\|fg\|_1 \leq \|f\|_p \|g\|_{p^*}$.
    \end{enumerate}
\end{proof}
\begin{proposición}[Desigualdad de Minkowski]
    Sea $p \in [1, \infty]$, $(X, \mu)$-espacio de medida y $f, g \in \mathcal{L}^p(X, \mathbb{F})$. Entonces 
    $$\|f + g\|_p \leq \|f\|_p + \|g\|_p$$    
\end{proposición}
\begin{proof}
    Distingamos casos: 
    \begin{enumerate}
        \item Si $\|p\|_p$ o $\|g\|_p = \infty$ el resulatdo es evidente
        \item Supongamos que $\|f\|_p, \|g\|_p < \infty$.
        Podemos ver que $\|f + g\|_p < \infty$, ya que 
        $$|f + g|^p \leq 2^{p-1}(|f|^p + |g|^p)$$
        Para demostrar este hecho, usamos que la función $h(x) = |x|^p$ es convexa sobre $\mathbb{R}^+$ (para $p > 1$) y por la definición de convexidad:
        $$\left|\frac{1}{2}f + \frac{1}{2}g\right|^p \leq \left|\frac{1}{2}|f| + \frac{1}{2}|g|\right|^p \leq \frac{1}{2}|f|^p + \frac{1}{2}|g|^p$$
        Esto implica que 
        $$|f + g|^p \leq 2^{p-1}|f|^p + 2^{p-1}|g|^p$$
        Por lo tanto, sabemos que $\|f + g\|_p < \infty$. Distingamos casos:
        \begin{enumerate}
            \item Si $\|f + g\|_p = 0$ el resultado es evidente. 
            \item Si $\|f + g\|_p \neq 0$, usando la desigualdad triangular y luego la desigualdad de Hölder, tenemos que 
\begin{align*}
\|f + g\|_p^p &= \int |f + g|^p \, \mathrm{d}\mu = \int |f + g| \cdot |f + g|^{p-1} \, \mathrm{d}\mu \\
&\le \int (|f| + |g|) |f + g|^{p-1} \, \mathrm{d}\mu \\
&= \int |f|\, |f + g|^{p-1} \, \mathrm{d}\mu + \int |g|\, |f + g|^{p-1} \, \mathrm{d}\mu \\
&\le \left( \int |f|^p \, \mathrm{d}\mu \right)^{\frac{1}{p}} \left( \int |f + g|^{p} \, \mathrm{d}\mu \right)^{1-\frac{1}{p}}
+ \left( \int |g|^p \, \mathrm{d}\mu \right)^{\frac{1}{p}} \left( \int |f + g|^{p} \, \mathrm{d}\mu \right)^{1-\frac{1}{p}} \\
&= (\|f\|_p + \|g\|_p) \|f + g\|_p^{p-1}.
\end{align*}
            Dividiendo por $\|f + g\|_p^{p-1}$ obtenemos la desigualdad deseada:
            $$\|f + g\|_p \leq \|f\|_p + \|g\|_p$$
        \end{enumerate}
    \end{enumerate}
\end{proof}
\begin{definición}[Espacio de Banach]
    Un espacio normado $(V, \|\cdot\|)$ se dice un espacio de Banach si es completo, es decir, si toda sucesión de Cauchy en $V$ converge a un elemento de $V$.
\end{definición}
\ejemplo{
    El espacio $\mathcal{B}(X, \mathbb{F})$ de funciones limitadas $f: X \to \mathbb{F}$ se dice que es un espacio de Banach con la norma $\|f\|_{\infty} = \sup_{x \in X} |f(x)|$.
}
\begin{proposición}
    $L^p(X, \mathbb{F})$ es un espacio de Banach para todo $p \in [1, \infty]$. Además, si $(f_n)_{n \in \mathbb{N}} \to f$ en $L^p(X, \mathbb{F})$, entonces existe una subsucesión $(f_{n_k})_{k \in \mathbb{N}}$ tal que $f_{n_k}(x) \to f(x)$ para casi todo $x \in X$.
\end{proposición}
\begin{proof}
    QUEDA PONERLA
\end{proof}
\begin{definición}[Espacio de sucesiones de potencia p-ésima sumables]
    Sea $p \in [1, \infty)$. Definimos $\ell^p = \{(x_n)_{n \in \mathbb{N}} : \|(x_n)_{n \in \mathbb{N}}\| \leq \infty\}$ donde 
    $$\|(x_n)_{n \in \mathbb{N}}\|_p = \left(\sum_{n = 1}^{\infty} |x_n|^p\right)^{1/p} \in \mathbb{R}^+$$
\end{definición}
\begin{observación}
    Observese que si $X = \mathbb{N}$, y $\mu$ es una medida de Cantor, es decir $\Sigma = 2^\mathbb{N}$ y $\mu(A) = \#(A \cap \mathbb{N})$ donde $\#$ denota el cardinal del conjunto si $A$ es finito y $\mu(A) = \infty$, en caso contrario, tenemos que $L^p(\mathbb{N}, \mu, \mathbb{F}) = \ell^p$.
\end{observación}
\begin{definición}[Continuidad absoluta]
    Sea $I$ un intervalo $(M, d)$ un espacio métrico, $f: I \to M$ se dice absolutamente continua si para todo $\varepsilon \in \mathbb{R}^+$ existe $\delta \in \mathbb{R}^+$ tal que para cualquier familia finita de intervalos abiertos contenidos en $I$, $\{(x_k, y_k)\}_{k = 1}^{n}$ que cumpla que $\sum_{k = 1}^{n} (y_k - x_k) z \delta$ se tiene que
    $$\sum_{k = 1}^{n} d(f(y_k), f(x_k)) < \varepsilon$$
    denotamos $\mathcal{AC}(I, M)$ al conjunto de las funciones absolutamente continua de $I$ en $M$. 
\end{definición}
\begin{observación}
    En caso de funciones del tipo $f = (f_1, \ldots, f_n): I \to \mathbb{R}^n$, $f$ es absolutamente continua si y sólo si $f_k$ es absolutamente continua para todo $k = 1, \ldots, n$
\end{observación}
\begin{definición}[Derivada débil]
    Sean $A \subset \mathbb{R}$ medible, $f, g: A \subset \mathbb{R} \to \mathbb{R}$ medibles. Decimos que $g$ es una derivada débil de $F$ si $g$ es Lebesgue-integrable y existen $c, x_0 \in \mathbb{R}$ tales que
    $$f(x) = c + \int_{A \cup [x_0, x)} g(y) dy$$
\end{definición}\
\begin{teorema}[Teorema Fundamental del Cálculo para la Integral de Lebesgue]
    Sea $f \in L^{1}([a, b], \mathbb{R})$. Entonces la función $F [a, b] to \mathbb{R}$ dada por 
    $$F(x) = c + \int_{a}^{x} f(y) dy$$
    donde $c \in \mathbb{R}$ es una constante cualquiera y derivable en casi todo punto $F^{\prime} = f$ en casi todo punto. Además, $F \in \mathcal{AC}([a, b], \mathbb{R})$ y toda función $F \in \mathcal{AC}([a, b], \mathbb{R})$ y de esta forma, en particular
    $$F(b) - F(a) = int_a^b dy \text{ (Regla de Barrow)}$$
\end{teorema}
\subsection{La relación entre las integrales de Riemann y de Lebesgue}
\begin{definición}[Integral de Riemann]
    Sea $f:[a,b]\to\mathbb{R}$ una función acotada. Una \emph{partición} de $[a,b]$ es un conjunto finito
    \[P=\{a=x_0<x_1<\dots<x_n=b\}.
    \]
    Para cada subintervalo $[x_{i-1},x_i]$ definimos
    \[M_i=\sup_{x\in[x_{i-1},x_i]} f(x),\qquad m_i=\inf_{x\in[x_{i-1},x_i]} f(x).
    \]
    La suma superior y la suma inferior de Riemann asociadas a la partición $P$ son
    \[U(f,P)=\sum_{i=1}^n M_i\,(x_i-x_{i-1}),\qquad L(f,P)=\sum_{i=1}^n m_i\,(x_i-x_{i-1}).
    \]
    La \emph{integral superior} (resp. \emph{integral inferior}) de $f$ en $[a,b]$ se definen por
    \[\overline{\int_a^b} f = \inf_P U(f,P),\qquad \underline{\int_a^b} f = \sup_P L(f,P),
    \]
    donde los ínfimos y supremos se toman sobre todas las particiones finitas $P$ de $[a,b]$. Decimos que $f$ es \emph{Riemann-integrable} en $[a,b]$ si
    \[\overline{\int_a^b} f = \underline{\int_a^b} f.
    \]
    En tal caso su valor común se llama integral de Riemann de $f$ en $[a,b]$ y se denota
    \[\int_a^b f(x)\,dx = \overline{\int_a^b} f = \underline{\int_a^b} f.
    \]

    Alternativamente, $f$ es Riemann-integrable con integral $I$ si y sólo si para todo $\varepsilon>0$ existe $\delta>0$ tal que para toda partición $P$ con norma $\|P\|:=\max_i (x_i-x_{i-1})<\delta$ y para cualquier elección de puntos marcados $t_i\in[x_{i-1},x_i]$ se cumple
    \[
        \left|\sum_{i=1}^n f(t_i)\,(x_i-x_{i-1}) - I\right| < \varepsilon.
    \]
\end{definición}
\begin{teorema}
    Sea una función limitada $f: [a, b] \subset \mathbb{R} \to \mathbb{R}$ es Riemann integral entonecs es Lebesgue integrable y ambas integables coinciden. 
\end{teorema}
\begin{proof}
Para cada $n \in \mathbb{N}$, dividimos el intervalo $I = [a, b]$ en subintervalos disjuntos
$$[a, a + 2^{-n}(b - a), \ldots, [b - 2^{-n}(b - a), b]$$
de longitud $2^{-n}$. Denotamos estos subintervalos por $I_{n,1}, \ldots, I_{n,2^n}$. Sean
$$m_{n,k} = \inf f(I_{n,k}), \quad M_{n,k} = \sup f(I_{n,k}).$$

Consideramos las funciones escalonadas
$$f_n(x) = m_{n,k} \text{ si } x \in I_{n,k}, \quad g_n(x) = M_{n,k} \text{ si } x \in I_{n,k}, \quad k = 1, \ldots, 2^n.$$

Es claro que
$$f_n(x) \leq f(x) \leq g_n(x).$$

Además,
$$f_n(x) \leq f_{n+1}(x), \quad g_{n+1}(x) \leq g_n(x).$$

Las funciones escalonadas son Riemann integrables y, en particular, $f_n$ satisface
$$\int_{[a,b]} f_n = 2^{-n} \sum_{k=0}^{2^n} m_{n,k}.$$

Estas funciones $f_n$ son también simples medibles y, por tanto, Lebesgue integrables, con
$$\int_{[a,b]} f_n = 2^{-n} \sum_{k=0}^{2^n} m_{n,k}$$

(se puede verificar). Así, $\int_{[a,b]} f_n = \int_{[a,b]} f_n$ y lo mismo ocurre con $g_n$. La integrabilidad Riemann de $f$ implica que 
$$\lim_{n \to \infty} \int_{[a,b]} f_n = \lim_{n \to \infty} \int_{[a,b]} g_n = \int_{[a,b]} f$$

Las funciones límite
\[\varphi = \lim_{n \to \infty} f_n, \quad \psi = \lim_{n \to \infty} g_n\]

son acotadas y medibles en el sentido de Lebesgue (pues son límite de funciones escalonadas), por lo que son integrables en el sentido de Lebesgue. Es claro que
para todo $n \in \mathbb{N}$. Por la igualdad \eqref{eq:2.2}, concluimos que

para todo $n \in \mathbb{N}$. Por la igualdad \eqref{2.2}, concluimos que
$$\int_{[a,b]} \varphi = \int_{[a,b]} \psi = \int_{[a,b]} f,$$

y dado que $\varphi(x) \leq \psi(x)$, tenemos que $\varphi(x) = \psi(x)$ casi en todas partes (se puede verificar). Como $\varphi \leq f \leq \psi$, se sigue que $f = \varphi = \psi$ casi en todas partes, por lo que $f$ es medible y
$$\int_{[a,b]} f = \int_{[a,b]} \varphi = \int_{[a,b]} \psi = \int_{[a,b]} f.$$
\end{proof}
\begin{corolario}
    Sea $f: [a, b] \to \mathbb{R}$ limitaa y sea $D$ el conjunto de dos puntos de discontinuidad de $f$ en $[a, b]$. Entonces $f$ es Riemann-integrable en $[a, b]$ si y sólo si $D$ es un conjunto de medida nula. 
\end{corolario}
\begin{proof}
Primero veamos que las funciones $\varphi$ y $\psi$ definidas en el apartado anterior a partir de la sucesión de particiones $\{P_n\}_{n\in\mathbb{N}}$ satisfacen que, para $c \in [a, b] \setminus \bigcup_{n\in\mathbb{N}} P_n$, se tiene que $\varphi(c) = \psi(c)$ si y solo si $f$ es continua en $c$.

Supongamos que $f$ es continua en $c$. Entonces, dado $\varepsilon > 0$, existe $\delta > 0$ tal que
$$|f(x) - f(c)| < \frac{\varepsilon}{2}.$$

Por consiguiente, dados dos elementos $x, y \in B(c, \delta)$, tenemos
$$|f(x) - f(y)| \leq |f(x) - f(c)| + |f(y) - f(c)| < \varepsilon.$$

En particular,
$$\sup\{f(x) : x \in B(c, \delta)\} - \inf\{f(x) : x \in B(c, \delta)\} \leq \varepsilon.$$

Consideremos entonces $N \in \mathbb{N}$ tal que $1/N < \delta$, es decir, $n \geq N$. En este caso, para $I_{n,j_n} \subset B(c, \delta)$, se deduce que
$$M_{n,j_n} - m_{n,j_n} \leq \sup\{f(x) : x \in B(c, \delta)\} - \inf\{f(x) : x \in B(c, \delta)\} \leq \varepsilon,$$

por lo que $|\varphi(c) - \psi(c)| \leq \varepsilon$, y como $\varepsilon \in \mathbb{R}^+$ es arbitrario, concluimos que $\varphi(c) = \psi(c)$.

Recíprocamente, supongamos que $\varphi(c) = \psi(c)$. Entonces, para $c \in I_{n,j_n}$, $M_{n,j_n} - m_{n,j_n} \to 0$ cuando $n \to \infty$. En este caso, dado $\varepsilon > 0$, existe $N \in \mathbb{N}$ tal que, para todo $n \geq N$,
$$M_{n,j_n} - m_{n,j_n} < \varepsilon.$$

En particular, dado $\delta > 0$ tal que $B(c, \delta) \subset I_{n,j_N}$, tenemos
$$|f(x) - f(c)| \leq M_{n,j_N} - m_{n,j_N} < \varepsilon, \quad \text{para todo } x \in B(c, \delta),$$

por lo que $f$ es continua en $c$.

Así, si $f$ es Riemann integrable en $[a, b]$, entonces existe un conjunto $E \subset [a, b] \setminus \bigcup_{n\in\mathbb{N}} P_n$ con $\lambda_1(E) = 0$ tal que
$$\varphi(x) = \psi(x) \quad \text{para todo } x \in \left([a, b] \setminus \bigcup_{n\in\mathbb{N}} P_n\right) \setminus E.$$

Denotamos
$$\widetilde{E} = E \cup \left(\bigcup_{n\in\mathbb{N}} P_n\right).$$

Como el conjunto $\bigcup_{n\in\mathbb{N}} P_n$ es numerable, tenemos que $\lambda_1(\widetilde{E}) = 0$. Por lo tanto, $f$ es continua en los puntos de $[a, b] \setminus \widetilde{E}$, es decir, $f$ es continua salvo en un conjunto de medida nula.

Recíprocamente, si $f$ es continua salvo en un conjunto de medida nula, entonces $\varphi(x) = \psi(x)$ para casi todo punto del intervalo $[a, b]$, de donde se deduce que las integrales superior e inferior coinciden, siendo entonces $f$ Riemann integrable en el intervalo $[a, b]$.
\end{proof}
\begin{teorema}
    Sea $I \subset \mathbb{R}$ un intervalo y $f: I \to \mathbb{R}$ tal que $f$ y $|f|$ son integrables en el sentido impropio de Riemann (i.e. en intervalos de integracion infinitos, con discontinuidades infinitas, no acotadas, etc). Entonces $f$ es integrable en el sentio de Lebesgue en $I$ y
    $$\int_R f = \int_L f$$
\end{teorema}
\begin{proof}
    Consideremos el caso en el que el intervalo es $I = (a, b]$ y limitado, y $f$ es integrable en el sentido propio de Riemann en caa intervalo $[a + \varepsilon, b]$ con $\varepsilon > 0$. O el caso en el que $a = -\infty$ es similar. \\
    El caso general se reduce a un número finito de los dos casos anteriores. Sea $f_n = f$ en $[a + n^{-1}, b]$ y $f_n = 0$ en $(a, a + n^{-1})$. Por la integrabilidad de Riemann la función $f$ es Lebesgue-medible en el intervalo $[a + n^{-1}, b]$ y en consecuencia $f_n$ es medible. Es claro que $f_n \to f$ puntualmente, por lo que $f$ es medible en $(a, b]$. \\
    Por la integrabilidad impropia de $|f|$ las funciones $|f_n| \leq |f|$ tienen integrales de Lebesgue uniformemente limitadas (iguales a sus correspondientes integrales de Riemann por el teorema anterior). Por el Lema de Fatou, la función $|f|$ es integrable-Lesbesgue.  Por el teorema de la convergencia dominada, las integrales de Lebesgue de las funciones $f_n$ sobre $(a,b]$  convergen a la integral de Lebesgue de $f$. \\
    En consecuencia, la integral de Lebesgue de $f$ coincide con la integral impropia de Riemann. 
\end{proof}
\subsection{Teoremas de Tonelli, Fubini y cambio de Variable}