
\section{Fundamentos de la teoría de la medida}
\subsection{Anillo, álgebra y $\sigma$-álgebra de conjuntos}
\begin{definición}[Anillo]
    Dados los conjuntos $X$ y $\mathcal{A} \subseteq \mathcal{P}(X)$, es decir una familia de subconjuntos no vacía, se dice que $\mathcal{A}$ es un \textbf{anillo de conjuntos} en $X$ si:
    \begin{enumerate}
        \item $\mathcal{A}$ es cerrado por uniones finitas, es decir, $\forall A, B \in \mathcal{A} \implies A \cup B \in \mathcal{A}$.
        \item $\mathcal{A}$ es cerrado por diferencias, es decir, $\forall A, B \in \mathcal{A} \implies A \setminus B \in \mathcal{A}$.
    \end{enumerate}
\end{definición}
\begin{definición}[Álgebra]
    Dado un conjunto $X$ y $\mathcal{A} \subseteq \mathcal{P}(X)$, familia de conjuntos no vacía, se dice que $\mathcal{A}$ es un \textbf{álgebra de conjuntos} en $X$ si:
    \begin{enumerate}
        \item $\mathcal{A}$ es cerrado por uniones finitas
        \item $\mathcal{A}$ es cerrado por complementos
        \item $X \in \mathcal{A}$
    \end{enumerate}
\end{definición}
\begin{definición}[$\sigma$-álgebra]
    Dado un conjunto $X$ y $\mathcal{A} \subseteq \mathcal{P}(X)$, familia de conjuntos no vacía, se dice que $\mathcal{A}$ es una \textbf{$\sigma$-álgebra de conjuntos} en $X$ si:
    \begin{enumerate}
        \item $\mathcal{A}$ es cerrado por uniones numerables
        \item $\mathcal{A}$ es cerrado por complementos
        \item $X \in \mathcal{A}$
    \end{enumerate}
\end{definición}
\begin{observación}
    Una álgebra es un anillo al que pertenece el conjunto total $X$. 
\end{observación}
\begin{observación}
    \vspace{-1cm}
    \begin{enumerate}
        \item Si $\mathcal{A}$ es un anillo, entonces tomando $A, B \in \mathcal{A}$, tenemos que $\emptyset = A \setminus A \in \mathcal{A}$ y $A \cap B = A \setminus (A \setminus B) \in \mathcal{A}$. Es decir, los anillos también son cerrados por intersección.
        \item Sea $\mathcal{A}$ anillo y $E \in \mathcal{A}$. Entonces $\mathcal{A}_E = \{A \in \mathcal{A} : A \subset E\} = \{A \cap E : A \in \mathcal{A}\}$ es una álgebra de conjuntos en $E$.
    \end{enumerate}
\end{observación}
\begin{definición}[Espacio medible]
    Dada una $\sigma$-algebra $\Sigma$ de un conjunto $X$ o también expresado como un par $(X, \Sigma)$ se llama espacio medible. A los conjuntos de $\Sigma$ se les llama conjuntos medibles. 
\end{definición}
\begin{definición}[Función medible]
    Dados dos espacios medibles $(X, \Sigma_X)$ y $(Y, \Sigma_Y)$, y una función $f: (X, \Sigma_X) \to (Y, \Sigma_Y)$, se dce que es medible si $\forall E \in \Sigma_Y f^{-1}(E) \in \Sigma_X$.
\end{definición}
\begin{lema}
    Sean un conjunto $X$, $\mathcal{C} \subset \mathcal{P}(X)$ y una familia $\mathfrak{U}$ de $\sigma$-algebras/álgebras/anillos en $X$ que contienen a $\mathcal{C}$. Entonces $\bigcap_{\mathcal{A} \in \mathfrak{U}} \mathcal{A}$ es una $\sigma$-álgebra/álgebra/anillo que llamamos \textbf{$\sigma$-álgebra/álgebra/anillo generada por $\mathcal{C}$} siendo la menor $\sigma$-álgebra/álgebra/anillo que contiene a $\mathcal{C}$.
\end{lema}

\subsection{Contenidos y medidas}
\begin{definición}[Contenido/Pre-medida]
    Sea $X$ un conjunto y $\Sigma$ un anillo en $X$. Se dice que una función $\mu: \Sigma \to [0, +\infty]$ es un contenido en $\Sigma$ si: 
    \begin{enumerate}
        \item $\mu(\emptyset) = 0$
        \item $\mu$ es aditivo, esto es que, dada una sucesión de finita de conjuntos $\{E_k\}_{k = 1}^{n} \subset \Sigma$ es tal que $E_k \cap E_j = \emptyset$ para $k \neq j$ entonces: 
        \[\mu\left(\bigcup_{k = 1}^{n} E_k\right) = \sum_{k = 1}^{n} \mu(E_k)\]
    \end{enumerate}
\end{definición}



\begin{definición}[Medida]
    Sea $X$ un conjunto y $\Sigma$-algebra en $X$. Se dice que una función $\mu: \Sigma \to [0, +\infty]$ es un contenido en $\Sigma$ si: 
    \begin{enumerate}
        \item $\mu(\emptyset) = 0$
        \item $\mu$ es $\sigma$-aditivo, esto es que, dada una sucesión numerable de conjuntos $\{E_k\}_{k = 1}^{\infty} \subset \Sigma$ es tal que $E_k \cap E_j = \emptyset$ para $k \neq j$ entonces: 
        \[\mu\left(\bigcup_{k = 1}^{\infty} E_k\right) = \sum_{k = 1}^{\infty} \mu(E_k)\]
    \end{enumerate}
\end{definición}
\begin{definición}[Espacio de medida]
    Dado un conjunto $X$, una $\sigma$-álgebra $\Sigma$ en $X$ y una medida $\mu: \Sigma \to [0, +\infty]$, se llama espacio de medida a la terna $(X, \Sigma, \mu)$.
\end{definición}
\begin{observación}
    En el caso particualar en el que $\mu(X) = 1$, se dice que $\mu$ es una \textbf{medida de probabilidad} para cada $E \in \Sigma$ y el espacio de medida se llama \textbf{espacio de probabilidad}.
\end{observación}
\begin{definición}[Espacio de medida finita]
    SE dice que $(X, \Sigma, \mu)$ es un espacio de medida finita si $\mu(X) < +\infty$.
\end{definición}
\begin{definición}[Espacio de medida $\sigma$-finita]
    SE dice que $(X, \Sigma, \mu)$ es un espacio de medida $\sigma$-finita si existe una sucesión numerable de conjuntos $\{E_k\}_{k = 1}^{\infty} \subset \Sigma$ tal que $X = \bigcup_{k = 1}^{\infty} E_k$ y $\mu(E_k) < +\infty$ para todo $k \in \mathbb{N}$.
\end{definición}
\begin{observación}
    Las definiciones anteriores se pueden aplicar de la misma manera a los contenidos.
\end{observación}
\begin{definición}[$\sigma$-álgebra de Borel]
    Teniendo en cuenta la definición formal de $\sigma$-álgebra, tenemos que $X$ es un espacio topológico con topología $\tau$ se dice que $\Sigma$ es una \textbf{$\sigma$-álgebra de Borel} con respecto $\tau$ y que $\mu$ es una medida de $\tau$-Borel o una medida de Borel con respecto a $\tau$, si $\tau \subset \Sigma$. 
    \\
    Decimos que la menor $\sigma$-álgebra que contiene a $\tau$ es la $\sigma$-álgebra de Borel de $\tau$ y la denotamos por $\mathcal{B}(\tau)$. 
    \\
    En el caso concreto en el que el espacio topológico sea $\mathbb{R}^n$ con la topología usual, denotaremos a esta $\sigma$-álgebra de Borel por $\mathcal{B}^n$.
\end{definición}
\begin{definición}[Función de Borel]
    Una función $f: (X, \tau_1) \to (Y, \tau_2)$ definida entre dos espacios topológicos se dice que es una \textbf{función de Borel} si $f: (X, \mathcal{B}(\tau_1)) \to (Y, \mathcal{B}(\tau_2))$ es medible.
\end{definición}
\begin{definición}[Medida de Borel regular]
    Dada una medida de Borel $\mu$ en un espacio topológico $(X, \tau)$, se dice que es una \textbf{medida de Borel regular} si es regular exterior, es decir, si cumple dos condiciones: 
    \begin{enumerate}
        \item Regularidad exterior: 
        $$\forall A \in \Sigma \mu(A) = inf\{\mu(U) : A \subset G, G \in \Sigma \text{ abierto}\}$$
        \item Regularidad interior:
        $$\forall A \in \Sigma \mu(A) = sup\{\mu(K) : F \subset A, F \in \Sigma \text{ compacto}\}$$
    \end{enumerate}
\end{definición}
\begin{definición}[Casi todo punto]
    Dado un espacio de medida $(X, \Sigma, \mu)$ y $E \in \Sigma$ diremos que una propiedad $P$ se cumple en $\mu$-casi todo punto de $E$ si existe $N \in \Sigma$, $N \subset E$ tal que $\mu(N) = 0$ y $P$ se cumple en $E/N$.
\end{definición}
\begin{definición}[Intervalo]
    Llamamos intervalos de $\mathbb{R}$ a los conjuntos conexos de $\mathbb{R}$ (contamos al vacío como intervalo). Decimos que $I \subset \mathbb{R}^n$ es un intervalo si $I = \prod_{i=1}^{n} I_i$ donde cada $I_i$ es un intervalo de $\mathbb{R} \forall i = 1, \ldots, n$.
\end{definición}
\begin{observación}
    Como la intersección de intervalos en $\mathbb{R}$ es un intervalo (la intersección de conexos es conexa), y dados dos intervalos $\prod_{i=1}^{n} I_i$ y $\prod_{i=1}^{n} J_i$ en $\mathbb{R}^n$, tenemos que:
    \[\left(\prod_{i = 1}^{n}I_i\right) \cap \left(\prod_{i = 1}^{n} J_i\right) = \prod_{i = 1}^{n} (I_i \cap J_i)\]
    es decir, la intersección de dos intervalos en $\mathbb{R}^n$ es un intervalo en $\mathbb{R}^n$. 
\end{observación}

\begin{lema}
    Sea $X$ un conjunto, $\mathcal{A}$ un anillo en $X$ y $\mu$ un contenido (medida) en $\mathcal{A}$. Entonces se cumplen las siguientes propiedades:
    \begin{enumerate}
        \item Monotonía y subaditividad: 
        $$\mu(A) \leq \mu(B) \forall A, B \in \mathcal{A}, A \subset B$$
        Si además, $\mu(B) < +\infty$ entonces $\mu(B \setminus A) = \mu(B) - \mu(A)$.
        \item Subaditividad de sucesión de conjuntos: Dado una sucesión de conjuntos $\{E_k\}_{k = 1}^{n} \subset \mathcal{A}$ entonces:
        $$\mu\left(\bigcup_{k = 1}^{n} E_k \leq \sum_{k = 1}^{n} \mu(E_k)\right)$$
        \item Sea $\{E_n\}_{n \in \mathbb{N}} \subset \mathcal{A}$, entonces
        $$\lim_{n \to \infty} \mu\left(\bigcup_{k = 1}^{n} E_k\right) \leq \sum_{k = 1}^{\infty} \mu(E_k)$$
        \item $\sigma$-subaditividad de medidas: Sea $\mu$ una medida, entonces
        $$\mu\left(\bigcup_{k = 1}^{\infty} E_k\right) \leq \sum_{k = 1}^{\infty} \mu(E_k)$$
        \item $\sigma$-superadtividad de medidas: Sea $\{E_n\}_{n \in \mathbb{N}} \subset \mathcal{A}$ es una familia disjunta dos a dos y además $\cup_{n \in \mathbb{N}} \in \mathcal{A}$, entonces
        \[\mu\left(\bigcup_{n \in \mathbb{N}} E_n\right) \geq \sum_{n = 1}^{\infty} \mu(E_n) = \lim_{n \to \infty} \mu\left(\bigcup_{k = 1}^{n} E_k\right)\]
    \end{enumerate}
\end{lema}
\begin{proposición}
    Sea $\mathcal{A}$ un anillo en $X$ y sea $\mu: \mathcal{A} \to [0, +\infty]$ un contenido. Entonces son equivalentes las siguintes condiciones:
    \begin{enumerate}
        \item $\mu$ es $\sigma$-aditiva, esto es: Dado una sucesión de conjuntos disjuntos dos a dos $\{B_n\}_{n \in \mathbb{N}}$ tal que $B = \cup_{n \in \mathbb{N}} B_n \in \mathcal{A}$ y $\mu(B) < +\infty$ entonces:
        $$\mu(B) = \sum_{n = 1}^{\infty} \mu(B_n)$$
        \item Continuidad en el $0$: Dada una sucesión de elementos $\{A_n\}_{n \in \mathbb{N}} \subset \mathcal{A}$ decreciente, es decir, $A_{n+1} \subset A_n$ para todo $n \in \mathbb{N}$ y tal que $\cap_{n \in \mathbb{N}} A_n = \emptyset$ y $\mu(A_1) < +\infty$, entonces:
        $$\lim_{n \to \infty} \mu(A_n) = 0$$
        \item $\mu$ es continua por debajo: Dada una sucesión de elementos $\{A_n\}_{n \in \mathbb{N}} \subset \mathcal{A}$ creciente, es decir, $A_n \subset A_{n+1}$ para todo $n \in \mathbb{N}$ y tal que $\mu(\cup_{n \in \mathbb{N}} A_n) < +\infty$, entonces:
        $$\lim_{n \to \infty} \mu(A_n) = \mu(A)$$
        \item $\mu$ es $\sigma$-subaditiva: Dada una sucesión de conjuntos $\{A_n\}_{n \in \mathbb{N}} \subset \mathcal{P}(X)$ y $A = \cup_{n \in \mathbb{N}} A_n \in \mathcal{A}$ y $\mu(A) < +\infty$, entonces:
        $$\mu(A) \leq \sum_{n = 1}^{\infty} \mu(A_n)$$
    \end{enumerate}
\end{proposición}
\begin{proof}
    (1) $\implies$ (2): Supongamos que $\mu$ satisface $(1)$ y que una sucesión $\{A_n\}_{n \in \mathbb{N}} \subset \mathcal{A}$ decreciente tal que $\mu(A_1) < +\infty$ y $\cap_{n \in \mathbb{N}} A_n = \emptyset$. Definimos $B_n = A_n \setminus A_{n+1}$, tales que pertenecen a $\mathcal{A}$, su unión es $A_1$ y son disjuntos dos a dos. Por lo tanto la sucesiñon dada por $s_n = \sum_{k = 1}^{n} \mu(B_k)$ es monotona creciente y esta limitada por $\mu(A_1) < \infty$, por lo que la serie $\sum_{k = 1}^{\infty} \mu(B_k)$ converge. \\
    Finalmente, tenemos que la cola de la serie $$\sum_{n = N}^{\infty} \mu(B_n) \xrightarrow{N \to \infty} 0$$
    Pero esta suma es igual a $\mu(A_N)$, ya que $\cup_{n = N}^{\infty} B_n = A_N$. Por lo tanto llegamos a la condición (2).\\
    (2) $\implies$ (3): \\
    Tomemos la sucesión de conjuntos $A_n = B \setminus B_n$ que satisface las hipótesis de (2) y por lo tanto 
    $$ 0 = \lim_{n \to \infty} \mu(A_n) = \lim_{n \to \infty} \mu(B \setminus B_n) = \mu(B) - \lim_{n \to \infty} \mu(B_n)$$
    por lo que $$\lim_{n \to \infty} \mu(B_n) = \mu(B)$$
    (3) $\implies$ (4): \\
    Tomando $B_n 0 \bigcup_{k = 1} ^{n} A_k$ que satisface las hipótesis de (3) y $$\mu(E) = \mu \left(\bigcup_{n \in \mathbb{N}} B_n \right) = \lim_{n \to \infty} \mu(B_n) = \lim_{n \to \infty} \mu\left(\bigcup_{k = 1}^{n} A_k\right) \leq \lim_{n \to \infty} \sum_{k = 1}^{n} \mu(A_k) = \sum_{k = 1}^{\infty} \mu(A_k)$$
    (4) $\implies$ (1): \\
    Basta aplicar el lema anterior para la demostración.
\end{proof}
\begin{lema}
    Sea $\mathcal{A}$ un anillo en $X$ y sea $\mu: \mathcal{A} \to [0, \infty]$. Si $\mu$ es aditiva y $\sigma$-subaditiva, entonces es $\sigma$-aditiva.
\end{lema}
\begin{proof}
    Se una sucesión de conjuntos $\{B_n\}_{n \in \mathbb{N}}$ disjuntos dos a dos tales que $B = \cup_{n \in \mathbb{N}} B_n \in \mathcal{A}$ y $\mu(B) < +\infty$. Entonces, tenemos que: 
    $$\sum_{k = 1}^{n} \mu (B_k) = \mu\left(\bigcup_{k = 1}^{n} B_k\right) \leq \mu(B) \leq \sum_{k = 1}^{\infty} \mu(B_k) \xrightarrow{n \to \infty}  \sum_{k = 1}^{\infty} \leq \mu(B) \leq \sum_{k = 1}^{\infty} \mu(B_k)$$
    Por lo tanto, $\mu(B) = \sum_{k = 1}^{\infty} \mu(B_k)$.
\end{proof}

\subsection{Medidas exteriores}
\begin{definición}[Medida exterior]
    Sea $X$ un conjunto. Dada una función $\mu: \mathcal{P}(X) \to [0, \infty]$ es una medida eterior si 
    \begin{enumerate}
        \item $\mu(\emptyset) = 0$
        \item $\mu(A) \leq \mu(B)$ si $A \subset B \subset X$
        \item $\mu$ es $\sigma$-subaditiva, es decir, dada una sucesión de conjuntos $\{A_n\}_{n \in \mathbb{N}} \subset \mathcal{P}(X)$ se cumple que:
        $$\mu\left(\bigcup_{n \in \mathbb{N}} A_n\right) \leq \sum_{n \in \mathbb{N}} \mu(A_n)$$
    \end{enumerate}
\end{definición}

\begin{definición}[Diferencia de dos conjuntos]
    Dado un conjunto $X$ y $A, B \subset X$ llamamos \textbf{diferencia} de $A$ y $B$ al conjunto $A \Delta B = (A \setminus B) \cup (B \setminus A)$.
\end{definición}
\begin{definición}[Medida exterior a partir de un contenido]
    Sea $X$ un conjunto $\mathcal{A} \subset \mathcal{P}(X)$ tal que $X \in \mathcal{A}$ y $\mu: \mathcal{A} \to [0, \infty]$. Definimos la función $\mu^*: \mathcal{P}(X) \to [0, \infty]$ asociada a $\mu$ tal que $\forall A \subset X:$
    $$\mu^*(A) = \inf\left\{\sum_{n \in \mathbb{N}} \mu(A_n) : \{A_n\}_{n \in \mathbb{N}} \subset \mathcal{A}, A \subset \bigcup_{n \in \mathbb{N}} A_n\right\}$$
\end{definición}
\begin{definición}[Conjunto medible]
    Sea $A$ un subconjunto de $X$ y $\mu^*$ una medida exterior en $X$. Se dice que $A$ es $\mu$-\textbf{medible} si: 
    $$\forall \epsilon > 0 \exists A_{\epsilon} \in \mathcal{A} : \mu^*(A \Delta A_{\epsilon}) < \epsilon$$
    Al conjunto de todos los conjuntos $\mu$-medibles se les denota por $\mathcal{A}_{\mu}$.
\end{definición}

\begin{lema}[Propiedades de una medida exterior (asociada a un contenido)]
    Sea $X$ un conjunto, $\mathcal{A} \subset \mathcal{P}(X)$ tal que $X \in \mathcal{A}, B, C \subset X$ y $\mu: \mathcal{A} \to [0, \infty]$-contenido, y $\mu^*$ la medida exterior asociada. Entonces: 
    \begin{enumerate}
        \item $\mu^*(B) \leq \mu^*(C)$ si $B \subset C$
        \item $\mu^*$ es $\sigma$-subaditiva, es decir, $\mu^*\left(\cup_{n = 1}^{\infty} A_n\right) \leq \sum_{n = 1}^{\infty} \mu^*(A_n)$
        \item $|\mu^*(B) - \mu^*(C)| \leq \mu^*(B \Delta C)$
        \item Si $\mathcal{A}$ es un anillo y $\mu$ es un contenido, entonces $\mu^*(\emptyset) = 0$ y equivalentemente, $\mu^*$ es una medida exterior en $X$.
    \end{enumerate} 
\end{lema}
\begin{proof}
    \begin{enumerate}
    \item $\mu^*$ es por definición monótona creciente, esto es, si $B \subset C \implies \mu^*(B) \leq \mu^*(C)$. Esto se debe a que cualquier recubrimiento numerable de $C$ es también un recubrimiento numerable de $B$.
    \item Sea una sucesión de conjuntos $\{A_n\}_{n \in \mathbb{N}} \subset \mathcal{A}$ tal que $A = \cup A_n$. Dado un $\epsilon > 0$, existen dos casos posibles, 
        \begin{enumerate}
            \item Existe $k \in \mathbb{N}$ tal que $\mu^*(A_k) = +\infty$. Entonces, como $A_k \subset A$ se tiene que $\mu^*(A) = \infty$ y por tanto no hay nada que demostrar.
            \item Si se cumple que $\forall n \in \mathbb{N} \mu^*(A_n) < \infty$ entonces, para cada una de las $n$ existe un recubrimiento $\{B_{n,k}\}_{k \in \mathbb{N}} \subset \mathcal{A}$ tal que $A_n \subset \cup_{k = 1}^{\infty} B_{n,k}$ y por tanto
            $$\sum_{k = 1}^{\infty} \mu(B_{n, k}) \leq \mu^*(A_n) + \frac{\epsilon}{2^n}$$
            Por lo tanto se tiene que $\cup A_n \subset \cup \cup B_{n, k}$ y en consecuencia se tiene que
            $$\mu^*\left(\bigcup_{n = 1}^{\infty} A_n\right) \leq \sum_{n = 1}^{\infty} \sum_{k = 1}^{\infty} \mu(B_{n, k}) \leq \sum_{n = 1}^{\infty} \mu^*\left(A_n\right) + \epsilon$$
            Como $\epsilon$ es arbitrario, tenemos el resultado. 
        \end{enumerate}
        \item Supongamos que $B \subset C \cup (B \Delta C)$, por lo que, por la subaditividad de $\mu^*$ obtenemos que: 
        $$\mu^*(B) \leq \mu^*(C) + \mu^*(B \Delta C) \iff \mu^*(B) - \mu^*(C) \leq \mu^*(B \Delta C)$$
        Intercambiando $B$ y $C$ obtenemos la otra desigualdad.
        \item Basta tomar $\{A_n\}_{n \in \mathbb{N}} = \{\emptyset\}$ como recubrimiento de $\emptyset$ y usar que como $\mu$ es un contenido, entonces $\mu(\emptyset) = 0$. El resto de propiedades ya han sido demostradas, dada cualquier funcion $\mu$ y en particular si $\mu$ es un contenido.
    \end{enumerate}
\end{proof}
\begin{teorema}
    Sea $X$ un conjunto, $\mathcal{A}$ una álgebra en $X$ y $\mu: \mathcal{A} \to [0, \infty]$ un contenido. Entonces: 
    \begin{enumerate}
        \item $\mathcal{A} \subset \mathcal{A}_{\mu} = \{E \in X : \forall \epsilon > 0 \exists A_{\epsilon} : \mu^*(E \Delta A_{\epsilon}) < \epsilon \}$ y la medida exterior $\mu^*$ coincide con $\mu$ en $\mathcal{A}$. 
        \item $\mathcal{A}_{\mu}$ es una $\sigma$-álgebra, y la restricción de $\mu^*$ a $\mathcal{A}_{\mu}$ es $\sigma$-aditiva
        \item La función $\mu^*$ es la única extensión $\sigma$-aditiva y positiva de $\mu$ en $\sigma$-álgebra generada por $\mathcal{A}$ y también es la única extensión $\sigma$-aditiva y positiva de $\mu$ a $\mathcal{A}_{\mu}$.
    \end{enumerate}
\end{teorema}
\begin{proof}
    %%TODO
\end{proof}
\begin{definición}[Pre-medida]
    Sea $X$ un conjunto, $\mathcal{A}$ un anillo en $X$ y $\mu: \mathcal{A} \to [0, \infty]$ un contenido. Se dice que $\mu$ es una pre-medida si es $\sigma$-aditiva.
\end{definición}
\begin{teorema}[Extension de un contenido a una medida]
    Sea $X$ un conjunto, $\mathcal{A}$ un anillo en $X$ y $\mu: \mathcal{A} \to [0, \infty]$ una pre-medida sobre un anillo $\sigma$-finito (dada $\{X_n\}_{n \in \mathbb{N}} \subset \mathcal{A}$ monótona creciente y con $\mu(X_n) < +\infty$ para todo $n \in \mathbb{N}$ y $X = \cup_{n \in \mathbb{N}} X_n$). \\\\
    Para cada $n$ denotaremos $\mu_{n} = \mu|_{\mathcal{A}_{n}}$ donde $\mathcal{A}_n = \{A \in \mathcal{A} : A \subset X_n\}$ (álgebra de conjuntos en $X_n$) y denotaremos por $\Sigma_n$ la $\sigma$-álgebra sobre $A$ sobre la cual $\mu^*$ es una medida exterior basada en $\mu$. Definimos: 
    $$\Sigma = \{A \subset X : A \cap X_n \in \Sigma_n \forall n \in \mathbb{N}\} \quad \overline{\mu}(A) = \lim_{n \to \infty} \mu_n(A \cap X_n) : A \in \Sigma$$ 
    Entonces $\Sigma$ es una $\sigma$-álgebra que no depende  de los $X_n$ escogidos, y $\overline{\mu}$ es la única medida en $\Sigma$ tal que $\overline{\mu}|_{\mathcal{A}} = \mu$.
\end{teorema}
\begin{proof}
    Cada una de las $\mu_n: A \to [0, \infty]$ es un contenido $\sigma$-aditivo y finito, por lo que por el teorema anterior podemos obtener las $\sigma$-algebras $\Sigma_n$ y las medidas $\mu^*_n |_{\Sigma_n}$. \\
    Sea $A_k$ tal que $A_k \cap X_n \in \Sigma_n$ para todo $n,k \in \mathbb{N}$. Por ser $\Sigma_n$ una $\sigma$-álgebra, tenemos que $(\cup_{k = 1}^{\infty} A_k) \cap X_n = \cup_{k = 1}^{\infty} (A_k \cap X_n) \in \Sigma_n$ para todo $n \in \mathbb{N}$ por lo que $\cup_{k = 1}^{\infty} A_k \in \Sigma$.
    %%TODO 
\end{proof}
\begin{definición}[Clase compacta]
    Sea una familia $\mathcal{K}$ de subconjuntos de un conjunto $X$, se dice que es una \textbf{clase compacta} si:
    $$\forall \{K\}_{n \in \mathbb{N}} : \bigcap_{n \in \mathbb{N}} K_n = \emptyset \exists N \in \mathbb{N} : \bigcap_{n = 1}^{N} K_n = \emptyset$$
\end{definición}
\ejemplo{
    Una familia arbitraria de conjuntos compactos de un espacio topologico es una clase compacta. En efecto parar demostrar el contrarecíproco tomemos una sucesion de compactos $\{K_n\}_{n \in \mathbb{N}}$ tal que $\cap_{n = 1}^{N} K_n = \emptyset \forall N \in \mathbb{N}$. Definimos $\overline{K_n} = \cup_{k = 1}^{n} K_k$. Entonces $\{\overline{K_n}\}_{n \in \mathbb{N}}$ es una sucesion encadenadad de conjuntos compactos y no vacíos, por lo que el teorema d ela interseccion compacta de Cantor, $\cap_{n \in \mathbb{N}} \overline{K_n} \neq \emptyset$ y como consecuencia $\cap_{n \in \mathbb{N}} K_n \neq \emptyset$.
}
\begin{teorema}
    Sea $\mu$ un contenido sobre $\mathcal{A}$ un anillo $\mathcal{A}$. Supongamos que existe una clase compacta $\mathcal{K}$ que aproxima a $\mu$ en el sentido de que: 
    Para todo $A \in \mathcal{A}$ con $\mu(A) < +\infty$ y para todo $\epsilon > 0$ existen $K_{\epsilon} \in \mathcal{K}$ y $A_{\epsilon} \in \mathcal{A}$ tales que
    $$A_{\epsilon} \subset K_{\epsilon} \subset A \quad \text{y} \quad \mu(A \setminus A_{\epsilon}) < \epsilon$$
    Entonces, $\mu$ es $\sigma$-aditiva. En particular esto es cierto si la la clase compacta $\mathcal{K}$ está contenida en $\mathcal{A}$ y para todo $A \in \mathcal{A}$ se cumple que 
    $$\mu(A) = \sup\{\mu(K) : K \subset A, K \in \mathcal{K}\}$$
\end{teorema}
\begin{proof}
    %%TODO
\end{proof}
\ejemplo{
    Consideremos los anillos 
}
\ejemplo{
    Consideremos el anillo $\mathcal{J}_0$ de uniones finitas de intervalos limitados de la forma $[a, b)$ en $\mathbb{R}$ y sea $g: \mathbb{R} \to \mathbb{R}$ una funcion creciente y continua por la izquierda, y sea $v_g[a, b) = g(b) - g(a)$, con $v_g$ un contenido sobre $\mathcal{J}_0$. Entonces, por el teorema anterior, llamese $\mu_g$ a la medida generada por el Teorema de la extensiones de medidas $\sigma$-finitas, se llama \textbf{medida de Lebesgue-Stieltjes} asociada a $g$.
}
\begin{definición}[Medida completa]
    Sea $\mathcal{A}$ una $\sigma$-álgebra en $X$ y $\mu: \mathcal{A} \to [0, +\infty]$ una medida. $\mu$ se dice completa si dado $E \in \mathcal{A}$ tal que $\mu(E) = 0$ entonces para cualquier $A \subset E \in \mathcal{A}$ $A$ es medible y $\mu(A) = 0$
\end{definición}
\begin{teorema}
    Las medidas dadas por el Teorema de Extensión de Medidas $\sigma$-finitas son completas.
\end{teorema}
\begin{teorema}
    Son equivalentes: 
    \begin{enumerate}
        \item $A$ es $\mu$-medible
        \item $\forall E \in \mathcal{A} : \mu^*(E) = \mu^*(E \cap A) + \mu^*(E \setminus A)$ (Condición de Carathéodory)
        \item $\forall \epsilon > 0 \exists B \in \mathcal{A} : \mu^*(A \Delta B) < \epsilon$ (Condición de aproximación)
    \end{enumerate}
\end{teorema}
\subsection{Propiedades de la medida de Lebesgue}
\subsubsection{Medida de Lebesgue y topología euclidiana}
\begin{definición}[Contenido de Jordan]
    Llamamos \textbf{contenido de Jordan} o \textbf{contenido de Peano-Jordan} a la función $\mu: \mathcal{J} \to [0, +\infty]$ definida como:
    $$\mu\left(\prod_{i=1}^{n} [a_i, b_i)\right) = \prod_{i=1}^{n} (b_i - a_i)$$
    donde, por simplicidad, entendemos que $0 \cdot (+\infty) = 0$. Para calcular el contenido de Jordan de un conjunto cualquiera $\mathcal{J}$, como es una unión finita de intervalos disjuntos de la forma $\prod_{i=1}^{b} [a_n, b_n)$, basta con sumar los contenidos de todos los intervalos que lo componen.
\end{definición}
\begin{definición}[Medida de Lebesgue]
Sea $\mathcal{J}_0$ el anillo de \emph{uniones finitas disjuntas de rectángulos limitados} de la forma 
\[
\prod_{k=1}^{n} [a_k, b_k) \subset \mathbb{R}^n,
\] 
y sea $\mu$ el \emph{contenido de Jordan} definido sobre $\mathcal{J}_0$.  

Aplicando el \emph{Teorema de Extensión de Medidas $\sigma$-finitas} (por ejemplo, tomando la secuencia $X_k = [-k, k]^n$ para $k \in \mathbb{N}$), se obtiene una única medida 
\[
\lambda_n : \mathcal{L}_n \to [0,\infty]
\] 
llamada \textbf{medida de Lebesgue} en $\mathbb{R}^n$, definida sobre la $\sigma$-álgebra $\mathcal{L}_n$ de \emph{conjuntos Lebesgue medibles}, que extiende el contenido de Jordan y es $\sigma$-aditiva.  

Además, para cualquier subconjunto $X \in \mathcal{L}_n$, podemos definir la \emph{medida restringida} 
\[
\lambda_X(A) := \lambda_n(A \cap X), \quad \forall A \in \mathcal{L}_n.
\]  
\end{definición}
\begin{definición}[Medida exterior de Lebesgue]
    Siguiendo el teorema anterior de obtención de una medida exterior a partir de un contenido sobre un anillo, en este caso el contenido de Jordan sobre el anillo $\mathcal{J}_0$ de uniones finitas de intervalos limitados de la forma $[a, b)$ en $\mathbb{R}$, obtenemos la \textbf{medida exterior de Lebesgue} $\lambda^*: \mathcal{P}(\mathbb{R}) \to [0, +\infty]$ definida como:
    $$\lambda^*(A) = \inf\left\{\sum_{n \in \mathbb{N}} \mu(A_n) : \{A_n\}_{n \in \mathbb{N}} \subset \mathcal{J}_0, A \subset \bigcup_{n \in \mathbb{N}} A_n\right\}$$
\end{definición}
\begin{definición}[Medida de Lebesgue-Stieltjes]
    Sea $g: \mathbb{R} \to \mathbb{R}$ una función creciente y continua por la izquierda. La \textbf{medida de Lebesgue-Stieltjes} asociada a $g$ es la medida de Lebesgue $\mu_g$ generada por el contenido definido en el anillo de uniones finitas de intervalos limitados de la forma $[a, b)$ como:
    $$v_g[a, b) = g(b) - g(a)$$

    
\end{definición}
\begin{lema}
    Todo intervalo $I = \prod_{k = 1}^{n} I_k \in \mathbb{R}^n$ es Lebesgue-medible. Si $I$ es degenerado, entonces $\lambda_n(I) = 0$. Si no, su medida coincide con el producto de las longitudes de sus aristas, es decir:
    $$\lambda_n(I) = \prod_{k = 1}^{n} (b_k - a_k)$$
\end{lema}
\begin{proof}
    Realicemos una distinción de casos: 
    \begin{enumerate}
        \item Si $I$ es degenerado, quiere decir que está contenido en un hiperplano paralelo a los ejes de coordenadas de $\mathbb{R}^n$, por lo que $I$ es medible y $\lambda_n(I) = 0$.
        \item Supongamos que $I$ es no-degenerado. Si $I = \prod_{k = 1}^{n} [a_k, b_k) \subset \mathbb{R}^n$ el resultado es evidente dada la definición de medida de lebesgue. 
        \item Consideremos ahora cualquier tipo de intervalo de la forma $I = \prod_{k = 1}^{n} I_k \subset\mathbb{R}^n$. Definamos $a_k = \inf{I_k}, b_k = \sup{I_k}$. Observese que $\partial I$ está contenida en la unión de $2n$ hiperplanos por lo que $\partial I$ es medible (por ser la medida de Lebesgue completa) y $\lambda_n(\partial I) = 0$. Ahora definamos $\mathcal{J} = \prod_{k = 1}^{n} [a_k, b_k)$ tenemos que $\mathcal{J}$ es medible y $\partial I$ es medible por lo tanto $I = (I \setminus \mathcal{J}) \cup (\mathcal{J}\setminus (\mathcal{J} \setminus I))$ es medible ya que $I \setminus \mathcal{J}, \mathcal{J} \setminus I \subset \partial I$. luego son medibles y se cumple que $$\lambda_n(I \setminus \mathcal{J}) = \lambda_n(\mathcal{J} \setminus I) = 0$$
        y además, 
        $$\lambda_n(I) = \lambda_n(I \setminus \mathcal{J}) + \lambda_n(\mathcal{J}) - \lambda_n(\mathcal{J} \setminus I) = \lambda_n(\mathcal{J}) = \prod_{k = 1}^{n} (b_k - a_k)$$    
    \end{enumerate}
\end{proof}

\begin{proposición}
    Todo conjunto abierto $U \subset \mathbb{R}^n$ es medible-Lebesgue. Es decir, $\lambda_n$ es una medida de Borel con respecto la topología usual, por tanto $\mathcal{B}(\mathbb{R}^n) \subset \mathcal{L}_n$.
\end{proposición}
\begin{proof}
    Todo abierto de $\mathbb{R}^n$ es una unión numerable de intervalos abiertos, y por tanto medible.
\end{proof}
\begin{proposición}
    La medida de Lebesgue en $\mathbb{R}^n$ es regula, es decir, para todo $A \in \mathcal{L}_n$ se cumple que:
    \begin{enumerate}
        \item Regularidad exterior: 
        $$\lambda_n(A) = \inf\{\lambda_n(U) : U \supset A, U \text{ abierto}\}$$
        \item Regularidad interior:
        $$\lambda_n(A) = \sup\{\lambda_n(K) : K \\subset A, K \text{ compacto}\}$$
    \end{enumerate}
\end{proposición}
\begin{proof}
    %%TODO
\end{proof}
\begin{definición}[Conjunto $G_{\delta}$]
    Un conjunto $A \subset \mathbb{R}^n$ es un conjunto $G_{\delta}$ si existe una sucesión numerable de conjuntos abiertos $\{U_k\}_{k \in \mathbb{N}}$ tal que:
    $$A = \bigcap_{k = 1}^{\infty} U_k$$
    
\end{definición}
\begin{definición}[Conjunto $F_{\sigma}$]
    Un conjunto $A \subset \mathbb{R}^n$ es un conjunto $F_{\sigma}$ si existe una sucesión numerable de conjuntos cerrados $\{F_k\}_{k \in \mathbb{N}}$ tal que:
    $$A = \bigcup_{k = 1}^{\infty} F_k$$
\end{definición}

\begin{observación}
    Como consecuencia del resultado anterior, concluimos que para todo conjunto Lebesgue medible $A \in \mathcal{L}_n$ existe un conjunto $U \in G_{\delta}$ y un conjunto $F \in F_{\sigma}$ tales que 
    $$F \subset A \subset U \quad \text{y} \quad \lambda_n(F) = \lambda_n(U) = \lambda_n(A)$$
\end{observación}
\subsubsection{Transformaciones de conjuntos medibles}
\begin{definición}[Aplicación lipschitziana]
    Sea $X, Y$ dos espacios métricos con métricas $d_X, d_Y$ respectivamente. Una función $f: X \to Y$ es \textbf{lipschitziana} si existe una constante $C > 0$ tal que:
    $$\forall x_1, x_2 \in X : d_Y(f(x_1), f(x_2)) \leq C d_X(x_1, x_2)$$
\end{definición}
\begin{teorema}
    Sea $F: (\mathbb{R}^n, ||\cdot||) \to (\mathbb{R}^m, ||\cdot||)$ una aplicación lipschitziana de constante $L \in \mathbb{R}^{+}$. Entonces para todo conjunto Lebesgue medible $A \subset \mathbb{R}^n$ el conjunto $F(A)$ también es Lebesgue medible y $\lambda_n(F(A)) \leq L^n\lambda_n(F(A))$
\end{teorema}
\begin{proof}
    Comencemos observando que si tenemos un hipercubo $I = \prod_{k = 1}^{n} [a_k, b_k) \subset \mathbb{R}^n$ de arista $r$ y de centro $y$, y $x \in I$ entonces
    $$\|F(x) - F(y)\|_{\infty} \leq L \|x - y\|_{\infty} \leq L \frac{r}{2}$$
    por lo que si $F(y) = (z_1, \ldots, z_n)$ entonces $F(x) \in \prod_{k = 1}^{n} [z_k - L\frac{r}{2}, z_k + L\frac{r}{2})$ y por lo tanto
    $$F(I) = F(\prod_{k = 1}^{n} [a_k, b_k)) \subset \prod_{k = 1}^{n} [z_k - L\frac{r}{2}, z_k + L\frac{r}{2})$$
    lo que implica que $\lambda_n(F(I)) \leq (Lr)^n = L^n \lambda_n(I)$. 
    \\Ahora, lo demostraremos para conjuntos medibles más generales. Por la regularidad interior de $\lambda_n(A) = \sup\{\lambda_n(K) : K \subset A, K \text{ compacto}\}$ podemos definir el conjunto 
    $$B = A \setminus \bigcup_{j = 1}^{\infty} K_j$$
    Entonces por construcción tenemos que $\lambda_n(B) = \lambda_n(A) - \lim_{j \to \infty} \lambda_n(K_j) = 0$, por tanto podemos deducir que  
    $$A = \bigcup_{j = 1}^{\infty} K_j \cup B$$
    donde cada $K_j$ se compacto y cada $B$ es un conjunto de medida nula. 
    Como $F$ es continua
    $$F\left(\bigcup_{j = 1}^{\infty} K_j\right) = \bigcup_{j = 1}^{\infty} F(K_j)$$
    y también es Borel-medible, por ser unión numerable de compactos. Así, se llega a demostrar que $F(B)$ también es medible. Sea $\varepsilon > 0$, podemos cubrir $B$ con una familia numerable de hipercubos de la formal
    $$I_j = \prod_{k = 1}^{n} [a_{j,k}, b_{j,k})$$
    de arista $r_j$ y centro $y_j = (y_{j, 1}, \ldots, y_{j,n})$ tales que $\sum_{j \in \mathbb{N}} \lambda_n(I_j) < \varepsilon$. TEnemos entones que 
    $$\sum_{j = 1}^{\infty} \lambda_n(F(I_j)) \leq \sum_{k = 1}^{\infty} L^n r^n_j = L^n \sum_{j = 1}^{\infty} r_{j}^n = L^n \lambda_n(I_j) < L^n \varepsilon$$
    Como $\varepsilon$ es arbitrario, se dedice que $F(B)$ tiene medida nula, y como la desigualdad se da para cualquier hipercubo, se tiene que
    $$\lambda_n(F(A)) \leq L^n \lambda_n(A)$$
\end{proof}
\begin{corolario}
    Sea $F: (\mathbb{R}^n, ||\cdot||) \to (\mathbb{R}^m, ||\cdot||)$ una aplicación bilipschitziana (función lipschitziana con inversa lipschitziana) de constante $L_2$ y $F^{-1}$ con constante $L_1$. Entonces para todo conjunto Lebesgue medible $A \subset \mathbb{R}^n$ el conjunto $F(A)$ también es Lebesgue medible y
    $$L_1^{n} \lambda_n(A) \leq \lambda_n(F(A)) \leq L_2^{n} \lambda_n(A)$$
\end{corolario}
\begin{corolario}
    Sea $F: (\mathbb{R}^n, ||\cdot||) \to (\mathbb{R}^m, ||\cdot||)$ una isometría para todo conjunto Lebesgue medible $A \subset \mathbb{R}^n$ el conjunto $F(A)$ también es Lebesgue medible y $\lambda_n(F(A)) = \lambda_n(A)$
\end{corolario}
\begin{teorema}
    Sea $A \subset \mathbb{R}^n$ un conjunto medible Lebesgue de medida finita. Entonces
    \begin{enumerate}
        \item $\lambda_n(A + h) = \lambda_n(A) \quad \forall h \in \mathbb{R}^n$
        \item $\lambda_n(U(A)) = \lambda_n(A) \quad \text{para todo operador lineal ortogonal } U: \mathbb{R}^n \to \mathbb{R}^n$
        \item $\lambda_n(cA) = |c|^n \lambda_n(A) \quad \forall c \in \mathbb{R}$
    \end{enumerate}
\end{teorema}
\begin{proof}
    %%TODO
\end{proof}